S\+DK \href{https://github.com/yangfuyuan/sdk/tree/big-screen}{\tt test} application for Y\+D\+L\+I\+D\+AR

Visit E\+AI Website for more details about \href{http://www.ydlidar.com/}{\tt Y\+D\+L\+I\+D\+AR} .

\section*{How to build Y\+D\+L\+I\+D\+AR S\+DK samples }

\$ git clone \href{https://github.com/yangfuyuan/sdk}{\tt https\+://github.\+com/yangfuyuan/sdk} \$ cd sdk \$ git checkout big-\/screen \$ cd .. \$ mkdir build \$ cd build \$ cmake ../sdk \$ make \#\#\#linux \$ vs open Project.\+sln \#\#\#windows

\section*{How to run Y\+D\+L\+I\+D\+AR S\+DK samples }

\$ cd samples

linux\+: \begin{DoxyVerb}$ ./ydlidar_test
$请输入端口号:/dev/ttyUSB0
$请输入雷达波特率:230400
\end{DoxyVerb}


windows\+: \begin{DoxyVerb}$ ydlidar_test.exe
$请输入端口号:COM3
$请输入雷达波特率:230400
\end{DoxyVerb}


You should see Y\+D\+L\+I\+D\+AR\textquotesingle{}s scan result in the console\+: \begin{DoxyVerb}Yd Lidar running correctly ! The health status: good
[YDLIDAR] Connection established in [/dev/ttyUSB0]:
Firmware version: 2.0.9
Hardware version: 2
Model: G4
Serial: 2018041400000102
[YDLIDAR INFO] Current Sampling Rate : 9K
[YDLIDAR INFO] Current Scan Frequency : 7.400000Hz
[YDLIDAR INFO] Now YDLIDAR is scanning ......
\end{DoxyVerb}


\section*{Upgrade Log }

2018-\/05-\/23 version\+:1.\+1.\+1

1.\+add auto connection for an exception.

2.\+add seril file lock.

2018-\/05-\/05 version\+:1.\+1.\+0

1.\+increase mouse event c++ mutil-\/platform library.

2.\+remove python mouse event.

3.\+add simple right mouse button, left button, middle button demo.

2018-\/05-\/02 version\+:1.\+0.\+0

1.\+Output screen corrdinates.

\section*{坐标系统 }

\href{https://github.com/yangfuyuan
    }{\tt } 

参数设定样例\+:

上图安装方式1\+: \begin{DoxyVerb}屏幕X轴最小值 = 0mm;

屏幕 Y 轴最小值 = 0mm;

屏幕X轴最大值 = 1920mm;

屏幕Y轴最大值 = 1080mm;

雷达安装位置X值 = 900mm;

雷达安装位置Y值 = -100mm;

雷达安装角度Theta = -90;//雷达坐标系相对屏幕坐标系逆时针旋转了90度.

CYdLidar laser;

laser.setMax_x(1920);

laser.setMax_y(1080);

laser.setMin_x(0);

laser.setMin_y(0);

LaserPose pose;

pose.x = 900;

pose.y = -100;

pose.theta = -90;

pose.reversion = false; //雷达表面朝外

laser.setpose(pose);
\end{DoxyVerb}


上图安装方式2\+: \begin{DoxyVerb}屏幕X轴最小值 = 0mm;

屏幕 Y 轴最小值 = 0mm;

屏幕X轴最大值 = 1920mm;

屏幕Y轴最大值 = 1080mm;

雷达安装位置X值 = -100mm;

雷达安装位置Y值 = 500mm;

雷达安装角度Theta = -180;//雷达坐标系相对屏幕坐标系逆时针旋转了180度.

CYdLidar laser;

laser.setMax_x(1920);

laser.setMax_y(1080);

laser.setMin_x(0);

laser.setMin_y(0);

LaserPose pose;

pose.x = -100;

pose.y = 500;

pose.theta = -180;

pose.reversion = false; //雷达表面朝外

laser.setpose(pose);
\end{DoxyVerb}


\paragraph*{备注\+:雷达安装位置和角度根据实际情况设定, 注意实际屏幕和操作屏幕的换算比例. 如果安装角度是120度, 直接把雷达安装角度设置为120, 不用更改雷达零点.}